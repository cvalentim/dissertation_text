\documentclass[12pt]{article}

\usepackage[brazilian]{babel}
\usepackage[utf8]{inputenc}

% This first part of the file is called the PREAMBLE. It includes
% customizations and command definitions. The preamble is everything
% between \documentclass and \begin{document}.

\usepackage[margin=1in]{geometry}  % set the margins to 1in on all sides
\usepackage{graphicx}              % to include figures
\usepackage{amsmath}               % great math stuff
\usepackage{amsfonts}              % for blackboard bold, etc
\usepackage{amsthm}                % better theorem environments
\usepackage{framed}
\usepackage{epstopdf}
\usepackage{multirow}

% various theorems, numbered by section

\newtheorem{thm}{Theorem}[section]
\newtheorem{lem}[thm]{Lema}
\newtheorem{prop}[thm]{Proposição}
\newtheorem{cor}[thm]{Corolario}
\newtheorem{conj}[thm]{Conjectura}
\newtheorem{defi}[thm]{Definição}

\DeclareMathOperator{\id}{id}

\newcommand{\bd}[1]{\mathbf{#1}}  % for bolding symbols
\newcommand{\RR}{\mathbb{R}}      % for Real numbers
\newcommand{\ZZ}{\mathbb{Z}}      % for Integers
\newcommand{\col}[1]{\left[\begin{matrix} #1 \end{matrix} \right]}
\newcommand{\comb}[2]{\binom{#1^2 + #2^2}{#1+#2}}


\begin{document}

\nocite{*}

\title{Estruturas de Dados para Séries Temporais}

\maketitle

\section{Introdução}
Blah Blah sobre motivação
\subsection{Definição do Problema}
\subsection{Resultados Obtidos}
\subsection{Organização da Dissertaçao}
 
\section{Conceitos Básicos}
\subsection{RMQ's 1-D}
\subsection{Orthogonal Range Queries}
\subsection{Trabalhos Relacionados}

\section{Estruturas de Dados para Detecção de Eventos}

Como explicado, nesse trabalho desenvolvemos estruturas
que permitem responder consultas para os problemas de \textit{todos os pares} e \textit{inícios}.
Como os dois problema são similares, muitas estratégias
para o primeiro tipo de problema funcionam para o segundo através
de simples adaptações. Sendo assim, vamos descrever e analisar
estratégias e conceitos sobre a pespectiva do problema de 
todos os pares e, em uma seção posterior, mostraremos como adaptar
as estratégias.

\subsection{Todos os pares}

Nessa seção apresentamos diferentes estruturas
para o problema \textit{todos os pares}, as estruturas descritas apresentam diferentes
características em termos de complexidade assintótica. Algumas delas
servem como baselines e ajudam a construir um melhor entendimento do problema.
A primeira estrutura mantém todas as diferenças entre os valores 
em listas ordenadas e responde consultas $(t, d)$ em $O(t + k)$. 
Essa é uma estrutura bastante eficiente para responder consultas,
porém a quantidade de informação armazenada $O(n^2)$ pode ser 
proibitiva na prática. A segunda estrutura se baseia no conceito de RMQ apresentado,
ela é eficiente em termos de espaço e tempo de pré-processamento
porém responde as consultas em tempo $O(n + k)$, ou seja,
sempre depende do tamanho da série de entrada. 

Por fim, descrevemos nossa estrutura. Ela se basea
no conceito de f-pairs e apresenta espaço e tempo esperado de 
pré-processamento de $O(n)$ e, além disso, responde as consultas em $O(k)$.

% Comentar abordagem gera : trabalhar com F-pairs em cima de estruturas consolidados como range trees, etc

\subsubsection{Lista Ordenada}

O pré-processamento consiste em montar $n$ listas ordenadas $L_1, L_2, \ldots, L_n$, 
essas listas mantém pares da série inicial. Para cada par ordenado da série de entrada $(a_i, a_j)$, com $i < j$ e $a_i < a_j$,
armazenamos a tupla $(a_j - a_i, i, j)$ na lista de índice $j - i$.
Após todos os pares serem inseridos as listas são ordenadas de forma decrescente. A Figura~\ref{fig:exemplolista} apresenta um exemplo.
Desta forma, para responder uma consulta $(t, d)$ simplesmente achá-se as soluções
dentro de cada lista $L_1, L_2, \ldots, L_t$. Fixada uma lista $L_i$ é simples
encontrar as soluções dentro dela, uma vez que os pares armazenados na mesma
estão ordenados de forma decrescente, basta começar a iterar do início
da lista e continuar enquanto os elementos tiverem valor maior ou igual a	$d$.
O pseudo-código na Figura~\ref{fig:listaquery} mostra o algoritmo. Uma observação importante é 
que o custo para achar as soluções dentro da lista $L_i$ é no máximo $k_{s_i} + 1$,
 onde $k_{s_i}$ é a quantidade de tuplas que correspondem a soluções dentro da $L_i$. 

% inserir figura
\begin{center}
\begin{figure}
\begin{framed}
Série de Entrada: $<3, 2, 5, 8>$ \\
$L_1: (3, 2, 3), (3, 3, 4)$ \\
$L_2: (6, 2, 4)$ \\
\end{framed}
\label{fig:exemplolista}
\caption{Índices começam por $1$}
\end{figure}
\end{center}



\begin{center}
\begin{figure}
\begin{framed}

{\bf ListaOrdenadaQuery}$(t, d)$

\hspace{1cm} {\bf For} $i \in [1, t]$

\hspace{2cm} $m \leftarrow 0$

\hspace{2cm} {\bf While} $m < |L_i|$ and $L_i[m][0] \ge d$

\hspace{3cm} Reporte o par $(L_i[m][1], L_i[m][2])$

\hspace{3cm} $m \leftarrow m + 1$
\end{framed}
\label{fig:listaquery}
\caption{Consulta}
\end{figure}
\end{center}

Com o mencionado, essa estratégia tem tempo de pré-processamento $O(n^2 \log n)$(ordenar uma lista de $n^2$ elementos) 
e tempo de pesquisa de no máximo $\sum_{i = 1}^{t} k_{s_i} + 1$, ou seja, $O(k + t)$,
onde $k$ é a quantidade de respostas a consulta $(t,d)$ feita.

\subsubsection{Estrutura baseada em RMQ}

Como apresentado na seção \ref{lala}, o RMQ é o problema de 
encontrar o mínimo (máximo) de um intervalo de vetor de forma 
eficiente. Como foi mostrado existem poderosas estruturas
de dados que permitem responder consultas RMQ em tempo $O(1)$, a partir de um pré-processamento $O(n)$. 
Nessa seção será apresentado um algoritmo que a partir de uma estrutura de RMQ para série
de entrada responde consultas $(t, d)$ em $O(n + k)$. 

A ideia principal do algoritmo é fixar um $i$ e encontrar todos os pares $(i, j)$ 
que são solução da pesquisa, ou seja, os pares em que $a_i$ é o menor elemento. 
Suponha que $a_m$ é o maior elemento dentro do intervalo $[i+1, i+2, \ldots, i+t]$, 
note que se $(i, m)$ não for uma solução para pesquisa $(t, d)$ (isto é, $a_m - a_i < d$) então
nenhum outro elemento do intervalo formará um par com $a_i$, por outro lado, se $a_m$  
forma solução com $a_i$, podemos reportar o par $(i, m)$ e seguir recursivamente explorando os intervalos
$[i+1, \ldots, m - 1]$ e $[m + 1, \ldots, t]$. Repare que, como a série foi
pré-processada e existe uma estrutura de RMQ para a mesma, o elemento $a_m$
pode ser encontrado em $O(1)$.

Achar soluções a partir de um elemento fixo é fundamental para a sequência
desse trabalho, sendo assim, o pseudo-código dessa etapa está em uma função separada chamada \verb|FindByFirst|,
apresentada na Figura~\ref{findbyfirst}. Repare que o pseudo-código assume que há acesso global a série
de entrada $A$ e que existe uma estrutura de RMQ para a mesma ($rmqA$). Com as mesmas
ideias aplicadas no \verb|FindByFirst|, podemos criar uma outra versão, o \verb|FindBySecond|, que
acha todas as soluções quando o maior elemento do par está fixado. O \verb|FindBySecond| está apresentado
nessa seção por completude, ele só será utilizado em uma seção posterior.

Finalmente, para de fato responder uma consulta, todas as soluções podem ser encontradas variando-se o 
elemento fixado da série e usando o \verb|FindByFirst| para encontrar as soluções com
o primeiro elemento fixado. A Figura~\ref{queryrmq} contém  o pseudo-código completo.

\begin{figure}
\begin{framed}
{\bf QueryRMQ}(t, d)

\hspace{1cm} {\bf For} $i \in [1, n]$:

\hspace{2cm} {\bf FindByFirst}(i, i, i + t, t, d):
\end{framed}
\label{queryrmq}
\caption{Responde a uma consulta $(t,d)$ a partir de uma estrutura de RMQ.}
\end{figure}

\clearpage
\begin{figure}
\begin{framed}
{\bf FindByFirst}(f, s, e, t, d)

\hspace{1cm} {\bf If} $e - s < 1$

\hspace{2cm} return

\hspace{1cm} $m \leftarrow rmqA.max(s, e)$

\hspace{1cm} {\bf If} $A[m] - A[f] < d$:

\hspace{2cm} return

\hspace{1cm} Reporte o par $(f, m)$

\hspace{1cm} {\bf FindByFirst}$(f, s, m - 1, t, d)$

\hspace{1cm} {\bf FindByFirst}$(f, m + 1, e, t, d)$

\end{framed}
\caption{Acha todas as soluções a partir de um início fixo. Assume que 
acesso global a série de entrada $A$ e a uma estrutura de RMQ para a mesma.}
\label{findbyfirst}
\end{figure}

\begin{figure}
\begin{framed}
{\bf FindBySecond}(f, s, e, t, d)

\hspace{1cm} {\bf If} $e - s < 1$

\hspace{2cm} return

\hspace{1cm} $m \leftarrow rmqA.min(s, e)$

\hspace{1cm} {\bf If} $A[f] - A[m] < d$:

\hspace{2cm} return

\hspace{1cm} Reporte o par $(m, f)$

\hspace{1cm} {\bf FindBySecond}$(f, s, m - 1, t, d)$

\hspace{1cm} {\bf FindBySecond}$(f, m + 1, e, t, d)$

\end{framed}
\caption{Acha todas as soluções a partir de um fim fixo. Assume que 
acesso global a série de entrada $A$ e a uma estrutura de RMQ para a mesma.}
\label{findbysecond}
\end{figure}


\begin{prop}
O procedimento {\bf FindByFirst} tem complexidade $O(k_i)$ em tempo,
onde $k_i$ é o número de pares da solução com $i$ como primeiro elemento.
\end{prop}  
\begin{proof}
A sequência de processamento do procedimento pode ser vista
como uma árvore estritamente binária, onde os nós internos são soluções e o único
custo extra é inerente as folhas da árvore. Como em toda
árvore estritamente binária a quantidade de folhas é igual a $\text{quantidade de nós internos} + 1$,
o resultado segue diretamente.
\end{proof} 

\begin{prop}
O procedimento {\bf QueryRMQ} tem complexidade de $O(n + k)$,
onde $k$ é a quantidade de pares que responde a consulta.
\end{prop}

\begin{proof}
Segue trivialmente com o resultado anterior.
\end{proof} 

\subsubsection{F-pairs}

Antes de apresentarmos a nossa abordagem precisamos definir 
o conceito de f-pairs. Os f-pairs são pares da sequência de entrada
com características especiais, definidas abaixo. Eles são importantes pois
a partir deles podemos responder qualquer consulta $(t, d)$ e,
como iremos mostrar, a quantidade esperada deles é $O(n)$. 
Abaixo definimos o que é um f-pair e apresentamos algumas de suas propriedades
em forma de lemas.

\begin{defi}
Um f-pair é um par $(i, j)$, com $i < j$, tal que $a_i < \min_{k \in [i + 1, j]} a_k$ e $a_j > \max_{k \in [i, j -1]} a_k$.
\end{defi}
Para deixar clara a definição, se a série de entrada for $<2, 10, 3, 5, 8>$ (índices começando em 1) os f-pairs
em relação à série é o conjunto $\{(1, 2); (3, 4); (3, 5); (4, 5) \}$. 

\begin{lem}
O valor esperado da quantidade de f-pairs em uma
permutação aleatória dos números $\{1, 2, \ldots, n\}$ é $n - H_n$, onde
$H_n$ é a soma dos $n$ primeiros termos da série harmônica.
\label{amountf}
\end{lem}

\begin{proof}
Seja a $F_{(i, j)}$ uma variável aleatória indicadora que assume valor
$1$ se $(i, j)$ é um f-pair e $0$, caso contrário. Nós estamos interessados
em encontrar $E[\sum_{i = 1}^{n-1} \sum_{j=i + 1}^n F_{(i, j)}]$, que, pela linearidade
do valor esperado é igual a $\sum_{i = 1}^{n-1} \sum_{j = i + 1}^n E[F_{(i, j)}]$.  

Como $F_{(i, j)}$ é uma variável indicadora seu valor esperado é $Pr[F_{(i,j)} == 1]$.
Para calcular esse valor devemos observar que para $(i, j)$ ser um f-pair, por definição,
$a_i$ deve ser o menor elemento do intervalo $[i, j]$ e $a_j$ o maior. Como estamos
tratando de permutações aleatórias, $a_i$ tem probabilidade de $\frac{1}{j - i + 1}$ de
ser o menor, e, uma vez que $a_i$ é o menor, $a_j$ tem probabilidade de $\frac{1}{j - i}$ de
ser o maior. Logo, $Pr[F_{(i, j)} = 1] = \frac{1}{(j - i + 1) (j - i)}$. Com isso podemos
estimar o somatório original:

$$\sum_{i = 1}^{n-1} \sum_{j = i + 1}^n  \frac{1}{(j - i + 1) (j - i)}$$

Extendendo a série telescópica, tem-se:

$$\sum_{i = 1}^{n-1} \sum_{j = i + 1}^n  \frac{1}{(j - i + 1) (j - i)} = \sum_{i = 1}^{n-1} \sum_{j = i + 1}^n  \frac{1}{(j - i)} - \frac{1}{(j - i + 1)} = \sum_{i = 1}^n 1 - \frac{1}{n - i + 1} = n - H_n $$

Com isso completamos a demonstração que o valor esperado do número de f-pairs é $n - H_n$.
\end{proof}

\begin{lem}
O resultado vale com alta probabilidade.
\end{lem}

\begin{proof}
Under construction.
\end{proof}

Note que o lema trata de permutações aleatórias, porém, qualquer série
pode ser mapeada em uma permutação com exatamente o mesmo número de f-pairs,
uma vez a definição de f-pair só utiliza o tamanho relativo dos elementos.
De forma concreta, seja uma série de entrada $<2.1, 1.14, 10.0, 0.1>$, 
esta série pode mapeada na permutação $<3, 2, 4, 1>$ com exatamente o mesmo número de
f-pairs.

Desta forma, o Lema~\ref{amountf} mostra que o número esperado de f-pairs não é grande
para qualquer possível série de entrada. Como será apresentado posteriormente os f-pairs podem ser usados para responder
consultas $(t, d)$, sendo assim, eles precisam ser gerados e armazenados
durante o pré-processamento. Desta forma, apresentamos o algoritmo na Figura~\ref{gemfpair} para geração
dos f-pairs. A idéia básica do pseudo-código é que para encontrar todos os f-pairs da
série $<a_1, a_2, \ldots, a_n>$ podemos fixar o menor elemento $a_m$ da série,
gerar todos os f-pairs envolvendo $a_m$, e depois continuar recursivamente para as partições $[1, m - 1]$ e $[m + 1, n]$.

\clearpage
\begin{figure}
\begin{framed}
{\bf GenFPairs}($A$, $e$, $s$)

\hspace{1cm} {\bf If} $e - s \le 1$

\hspace{2cm} return

\hspace{1cm} $m \leftarrow rmqA.min(s, e)$

\hspace{1cm} $t \leftarrow s$

\hspace{1cm} {\bf While} $t \ge m + 1$:

\hspace{2cm} $t \leftarrow rmqA.min(m + 1, t)$

\hspace{2cm} {\bf If} $A[m] - A[t] < d$:

\hspace{3cm} break

\hspace{2cm} Reporte o f-par $(m, t)$

\hspace{2cm} $t \leftarrow t - 1$

\hspace{1cm} {\bf GenFPairs}$(A, s, m - 1)$

\hspace{1cm} {\bf GenFPairs}$(A, m + 1, e)$

\end{framed}
\caption{Gera todos os f-pairs na presentes na sequência $A$.}
\label{gemfpair}
\end{figure}

\begin{prop}
O algoritmo {\bf GenFPairs} tem complexidade de tempo linear na quantidade
de f-pairs da série da A.
\end{prop}

Finalmente mostraremos como responder consultas $(t, d)$ a partir dos f-pairs.
O resultado abaixo mostra a relação dos f-pairs com as respostas de consultas.

\begin{lem}
\label{fundamental}
Seja $(i, j)$ uma saída para uma consulta $(t, d)$. Então, temos o seguinte:
\begin{enumerate}

\item Existe um índice $i*$ tal que $(i*, j)$ é uma solução e $i*$ é o começo
de um f-pair que é também uma solução.

\item Existe um índice $j*$ tal que $(i, j*)$ é uma solução e $j*$ é o final 
de um f-pair que também é uma solução.

\end{enumerate}
\end{lem}

\begin{proof}
Vamos apenas provar a primeira condição, a segunda é análoga. Seja $i*$ o índice
de menor entre os índices do conjunto ${i, \ldots, j-1}$. Em caso de empate,
nós consideramos o maior índice. Como $a_{i*} < a_i$ e $(i, j)$ é um evento, nós
temos que $(i*, j)$ também é um evento. Seja $k*$ o índice de maior valor em A
entre aqueles do intervalo ${i* + 1, \ldots, j}$. O par $(i*, k*)$ é um evento
e um f-pair. O que completa a demonstração.
\end{proof}



Com o Lema~\ref{fundamental}, podemos usar a seguinte estratégia 
para encontrar todas as soluções para uma pesquisa $(t,d)$:

\begin{enumerate}
\item Encontrar todos os f-pairs que satisfazem a consulta 
\item A partir dos f-pairs encontrados, gerar todas as soluções.
\end{enumerate}

Vamos nos concentrar primeiro na segunda parte, ou seja, gerar as soluções
a partir dos f-pairs. Sendo assim, podemos assumir que temos 
$F_{(t,d)}$, o conjunto de f-pairs que são solução para
consulta $(t,d)$. 

Sejam os conjunto $I_f = \{s_1, s_2, \ldots, s_{m_s}\}$
e $E_f = \{e_1, e_2, \ldots, e_{m_e}\}$, respectivamente, o conjunto de inícios e fins distintos
dos f-pairs em $F_{(t,d)}$. Para cada início em $I_f$ usa-se o algoritmo \verb|FindByFirst| (Figura~\ref{findbyfirst})
apresentado para encontrar todas as suas soluções.  Note que ao final desse processamento, todas
as soluções dos tipos $1$ e $3$ (de acordo com o lema \ref{fundamental}) já foram geradas. Para terminar
basta encontrar as soluções do tipo $2$. Fazemos isso  a partir conjunto $E_f$, usando o algoritmo
\verb|FindBySecond| modificado, onde, antes de gerar cada solução, ele verifica em um \textit{hash} se a solução
já foi gerada. O hash neste caso é simples, uma vez que os inícios são processados primeiro, todos os
f-pairs que têm como início um elemento do conjunto $I_f$ já foram gerados, sendo assim, no \verb|FindBySecond|,
para saber se uma solução já foi gerada, basta verificar se o início dela faz parte do conjunto $I_f$, ou seja,
o hash é apenas um vetor com os inicíos em $I_f$ marcados.

Note que da forma descrita, uma vez que temos $I_f$ e $E_f$, o número de passos para encontrar todas as soluções é $O(k)$, onde $k$ é o número de respostas a consulta.
Além disso, note que uma solução é processada no máximo duas vezes (uma a partir do conjunto $I_f$ e outra pelo conjunto $E_f$).
Na Figura~\ref{fpairqueryphase2} há um pseudo-código dessa etapa.

\begin{figure}
\begin{framed}
{\bf FPairQuery\_Phase2}(A, $I_f$, $E_f$, $t$, $d$)

\hspace{1cm} {\bf For} $inic \in I_f$

\hspace{2cm} {\bf FindByStart}(inic, inic, inic + t)

\hspace{1cm} {\bf For} $fim \in E_f$

\hspace{2cm} {\bf FindByEnd}(fim, fim - t, fim)

\end{framed}
\caption{Acha todas as soluções a partir dos inícios e fins dos f-pairs.}
\label{fpairqueryphase2}
\end{figure}


Nesse contexto, tudo o que falta para responder uma pesquisa $(t,d)$
é encontrar os f-pairs que também são solução para a pesquisa. 
Para realizar esse passo de forma eficiente os f-pairs devem
estar dispostos de maneira adequada em um vetor de tuplas da forma $(a_j - a_i, i, j)$,
onde $(i, j)$ é um f-pair.

Precisamos de um pouco de notação para deixar clara a disposição dos f-pairs.
Chamaremos de $F_t$ o conjunto de f-pairs $(i, j)$ tal que $j - i = t$. Agora 
a disposição: Os primeiros $|F_1|$ elementos do vetor devem se tuplas que correspondem aos f-pairs em $F_1$,
os próximos $|F_2|$ elementos tuplas que correspondem aos f-pairs em $F_2$ e assim por diante 
até os últimos elementos do vetor, que devem corresponder aos elementos em $F_n$.
Depois do vetor criado da forma estabelecida (note que é fácil modificar o 
algoritmo de geração de f-pairs para gerar um vetor dessa forma) uma estrutura de
RMQ deve ser criada para o vetor e as somas parciais dos valores $|F_1|, |F_2|, \ldots, |F_n|$
devem ser armazenadas em outro vetor. 

A organização descrita dos elementos é útil pois os f-pairs que podem responder uma consulta $(t,d)$ 
estarão aglomerados nos $\sum_{i = 1}^t |F_i|$ primeiros elementos do vetor. Sendo assim,
podemos, a partir da estrutura de RMQ, descobrir qual é o maior f-pair
$f_m$ entre $[1, \sum_{i = 1}^t |F_i|]$ (note que o maior f-pair $(i, j)$ é o que maximiza $a_j - a_i$).
Se o maior elemento for menor que $d$ podemos parar a pesquisa, não há f-pairs que satisfaçam a consulta. Caso contrário, reportamos
o par $f_m$ e processamos recursivamente $[1, m - 1]$ e $[m + 1, \sum_{i=1}^t |F_i|]$.
O pseudo-codigo na Figura~\ref{fpairqueryphase1} mostra o procedimento para encontrar
os f-pairs desejados e, por fim, a Figura~\ref{fpairquery} mostra o código completo
para responder a consulta.

De forma resumida, abaixo temos os componentes que consistuem a estrutura e
o espaço utilizado por cada um deles, assumindo que a implementação de rmq será a \verb|rmqBucket|.

\begin{center}
\begin{itemize}
\item Série de entrada, espaço $n$.
\item Vetor com pares $(a_j - a_i, i)$ para cada f-pair $(i, j)$, espaço $2f$, onde $f$ é o total de f-pairs.
\item Vetor de tamanho $n$ que guarda as somas parciais de $|F_1|, |F_2|, \ldots, |F_n|$, espaço $n$.
\item Estrutura de RMQ para o vetor de pares $(a_j - a_i, i)$, espaço $8f$.
\item Duas estruturas de RMQ (uma de mínimo e outra de máximo) sobre os valores da série de entrada, espaço $2\times4n$.
\item Dois vetores de tamanho $n$ usados como $hash$ para os inícios e fins já vistos durante uma consulta, espaço $2n$.
\end{itemize}
\end{center}

Desta forma, o espaço total consumido pela estrutura é $n + 2f + n + 8f + 8n + 2n = 10f + 11n$. Sendo assim, 
o espaço esperado é $22n$.


\clearpage
\begin{figure}
\begin{framed}
{\bf FPairQuery\_Phase1}(A, $s$, $e$, $t$, $d$)

\hspace{1cm} {\bf If} $e - s < 0$

\hspace{2cm} return

\hspace{1cm} $m \leftarrow rmqFpair.max(s, e)$

\hspace{1cm} {\bf If} $A[m] < d$:

\hspace{2cm} return

\hspace{1cm} $I_f \leftarrow I_f + F[m].inic$

\hspace{1cm} $E_f \leftarrow E_f + F[m].fim$

\hspace{1cm} {\bf FPairQuery\_Phase1}(A, $s$, $m - 1$, $t$, $d$)

\hspace{1cm} {\bf FPairQuery\_Phase1}(A, $m + 1$, $e$, $t$, $d$)

\end{framed}
\caption{Acha todos os f-pairs que são solução para uma pesquisa $(t, d)$.}
\label{fpairqueryphase1}
\end{figure}


\begin{figure}
\begin{framed}
{\bf FPairQuery}(A, $s$, $e$, $t$, $d$)

\hspace{1cm} $I_f, E_f \leftarrow$ {\bf FPairQuery\_Phase1}($A$, $s$, $e$, $t$, $d$)

\hspace{1cm} {\bf FPairQuery\_Phase2}($A$, $I_f$, $E_f$)
\end{framed}
\caption{Responde uma pesquisa $(t, d)$ usando o conceito de f-pairs.}
\label{fpairquery}
\end{figure}

\begin{prop}
O algoritmo {\bf FPairQuery} tem complexidade de tempo $O(k)$.
\end{prop}

\subsection{Inícios}

Nessa seção apresentamos as estruturas desenvolvidas para o problema \textit{inicios}.
Relembrando que essa problema é parecido com o \textit{todos os pares}, exceto que a
resposta para uma consulta são apenas os inícios dos pares que respondem a mesma. Isso
muda significatemente o problema, uma vez que no primeiro problema, \textit{todos os pares},
o tamanho da saída varia de $0$ até $n(n-1)/2 = O(n^2)$, enquanto neste varia de $0$ até $n$.

A primeira estrutura que apresentamos é uma variação bem simples da estrutura baseada
em RMQ descrita para o problema de todos os pares, ela tem tempo e espaço de pré-processamento
linear, se utilizada uma estrutura de RMQ adequada, e tempo de consulta $\theta(n)$. Já
a segunda estrutura apresentada se basea no conceito de f-pairs.

\subsection{Estrutura Baseada em RMQ}

O pré-processamento dessa estratégia consiste em criar uma estrutura de RMQ (máximo)
para a série. Para responder uma consulta $(t, d)$, simplesmente iteramos sobre todos
os inícios possíveis e verificamos em $O(1)$ se ele é ou não uma resposta para consulta.
Para fazer isso, basta notar que fixado um início $i$, podemos encontrar o elemento de maior valor
no intervalo $[i + 1, i + t]$ e verificar se ele forma um par com elemento $i$.
 Como a existe uma estrutura de RMQ para série, o elemento de maior valor
pode ser encontrado em $O(1)$.


\subsection{Estrutura Baseada em F-Pairs}

Essa estrutura busca extender a ideia dos f-pairs para o problema dos inícios.
O pré-processamento consiste em gerar todos os f-pairs da série e armazená-los
em uma Orthogonal Range Tree. O processo para gerar de forma eficiente os f-pairs é o mesmo
descrito para o problema de todos os pares na Figura~\ref{gemfpair}. Para cada 
f-pair $(i, j)$ criamos uma tupla $(j, a_j - a_i, j - i)$ na Orthogonal Range Tree.

No processo de resposta a consulta, recuperamos o conjunto $E_f$, o mesmo
conjunto definido na seção~\ref{lala}, isto é, o conjunto dos fins de f-pairs que
respondem a consulta $(t, d)$ dada. Para encontrar o conjunto $E_f$ fazemos consultas a
Range Tree. A primeira consulta é dá forma $x_1 \ge 0,  x_2 \ge d,  x_3 \ge 0, x_3 \le t$, se não
houver resposta para essa consulta significa que não há f-pair que satisfaça a consulta $(t,d)$.
Por outro lado, se existir uma solução, digamos, uma tupla $(j, \delta d, \delta t)$, então
podemos adicionar o fim $j$ ao conjunto $E_f$, e, recursivamente, explorar as regiões $[0, j - 1]$ e $[j + 1, n]$.
Isto é feito novamente com as pesquisas $x_1 \ge 0, x_1 \le j, x_2 \ge d, x_3 \ge 0, x_3 \le t$ e 
$x_1 > j, x_1 \le n, x_2 \ge d, x_3 \ge 0, x_3 \le t$ e, para cada resultado dessas pesquisas, repete-se
o mesmo processo. Note que a cada pesquisa, como limitamos a primeira coordenada de forma adequada, um 
mesmo fim não se repete. Na Figura~\ref{fpairinicqueryphase1} apresentamos o pseudo-código para esse procedimento.
A complexidade de tempo desta fase é $|E_f| \log^2 |E_f|$.

\begin{figure}
\begin{framed}
{\bf FPairInicQuery\_Phase1}(RS, s, e)

\hspace{1cm} {\bf If } $s > e$:

\hspace{2cm} {\bf Return}

\hspace{1cm} $(j, \delta d, \delta t) \leftarrow $ RS.query($x_1 \ge s,  x_1 \le e,  x_2 \ge d,  x_3 \ge 0, x_3 \le t$)

\hspace{1cm} {\bf If} $(j, \delta d, \delta t)$ is NULL

\hspace{2cm} {\bf Return}

\hspace{1cm} $E_f \leftarrow E_f + j$

\hspace{1cm} {\bf FPairInicQuery}(RS, $s$, $j - 1$)

\hspace{1cm} {\bf FPairInicQuery}(RS, $j + 1$, $e$)

\end{framed}
\caption{Acha o conjunto de fins $E_f$.}
\label{fpairinicqueryphase1}
\end{figure}




Com o conjunto de fins $E_f$, precisamos gerar os inícios para de fato responder as consultas.
Uma forma simples de gerar os inícios seria, para cada fim em $E_f$, utilizar o algoritmo \verb|FindBySecond| para 
gerar os pares que são solução para consulta, e, para cada par $(i, j)$ gerado,
verificar em um \textit{hash} se o início $i$ já foi reportado, e, caso não tenha sido, 
reportá-lo e marcá-lo na hash. O problema desse algoritmo é que ele irá visitar todos
pares que são solução para pesquisa $(t, d)$ e, como a quantidade total desses pares 
pode ser quadrática no tamanho da série, o algoritmo não é eficiente. Abaixo
descrevemos duas formam que evitam processar um mesmo início mais de uma vez.

Uma forma mais eficiente de reportar os inícios a partir do conjunto $E_f$ é 
ordená-los de forma decrescente por valor e processá-los nessa ordem. Suponha que encontramos
todos os inícios que foram par com o fim $i$, e agora queremos encontrar
os inícios que formam par com um fim $j$ de valor menor que $i$ ($e_i > e_J$),
uma observação importante é que não precisamos buscar na região $[i - t, i]$, pois,
se algum início dessa região formar um par com $j$ ele também forma um par com $i$, e,
sendo assim, já foi reportado. De forma geral, se os fins forem processados em 
ordem decrescente por valor, para cada novo fim na iteração só precisamos buscar novos
inícios em regiões que não foram cobertas por fins processados antes. Na Figura~\ref{fpairinicqueryphase2}
há o pseudo-código dessa etapa. 

\begin{figure}
\begin{framed}
{\bf FPairInicQuery\_Phase2}(A, $E_f$)

\hspace{1cm} Ordene $E_f$ de forma descrescente por valor

\hspace{1cm} Crie uma árvore binária de busca BST

\hspace{1cm} \textbf{For} $fim \in E_f$

\hspace{2cm} prev $\leftarrow $ BST.prev(fim)

\hspace{2cm} suc $\leftarrow $ BST.suc(fim)

\hspace{2cm} BST.add(fim)

\hspace{2cm} $s \leftarrow \max\{fim - t, \text{prev}\}$ 

\hspace{2cm} $e \leftarrow \min\{fim, \text{suc} - t \}$ 

\hspace{2cm} {\bf FindBySecond}$(f, s, e, t, d)$

\end{framed}
\caption{Acha todos os inicios a partir do conjunto de fins $E_f$ .}
\label{fpairinicqueryphase2}
\end{figure}

Note que o algoritmo acima processa cada início exatamente uma vez.
Sendo assim, o total de operações realizadas por chamadas ao {\bf FindBySecond}
é limitado pelo número de ínicios. Por outro lado, para cada fim em $E_f$ fazemos
duas buscas e uma inserção em um árvore binária de busca, se está árvore for balanceada (AVL, RedBlack, etc.),
 o custo de cada uma dessas operações é $\theta(\log n)$. Desta forma, o tempo gasto por esta fase é da ordem $O(k + |E_f|\log |E_f|)$,
 onde $k$ é o total de respostas (inícios) a consulta $(t, d)$ feita.

Contudo, existe outra forma mais eficiente de recuperar os inícios a partir do conjunto
$E_f$. Se ordenarmos o conjunto $E_f$ de forma decrescente por índice, ou seja, fins mais a direita
ficam na frente do vetor, podemos achar os .....

%\begin{figure}
%\begin{framed}
%{\bf FPairInicQuery\_Phase2}(A, $E_f$)

%\hspace{1cm} Ordene $E_f$ de forma decrescente por índice

%\hspace{1cm} Crie uma pilha P 

%\hspace{1cm} {\bf For} $fim \in E_f$

%\hspace{2cm} {\bf If} P é vazia

%\hspace{3cm}  P.push((A[f], fim, fim -t))

%\hspace{3cm}  continue

%\hspace{2cm}  {\bf If} P.top()[0] < A[fim]

%\hspace{3cm}  (f, e, s) $\leftarrow$ P.pop()

%\hspace{3cm}  \verb|FindByEnd|(f, e, s)
%
%\hspace{3cm}  P.push((fim, fim - t, 

%\end{framed}
%\caption{Acha todos os inicios a partir do conjunto de fins $E_f$ .}
%\label{fpairinicqueryphase2}
%\end{figure}


\subsection{Extensões para outras consultas}
 Explicar nossa abordagem e outras triviais

\section{Avaliação Experimental}

Nessa seção buscamos avaliar as estruturas propostas sobre o enfoque experimental.
 Mantendo a linha das outras seções, iremos primeiros descrever os experimentos para o problema de \textit{todos os pares}
e então seguir para o problema \textit{inícios}.

\subsection{Corpus}

Dois conjuntos de séries foram utilizadas durante os experimentos. 
O primeiro conjunto é formado por $5$ permutações geradas aleatoriamente, de tamanhos $1$MB, $2$MB, $3$MB, $4$MB, $5$MB.
Já o segundo conjunto é formado por $6$ séries reais advindas do mercado financeiro,
 com tamanhos especificados na Figura~\ref{tamanho-series}.

%\begin{figure}
%\begin{center}
%\begin{tabular}{|c|c|c|c|c|c|c|}
%\hline
%    & \textbf{Série 1} & \textbf{Série 2} & \textbf{Série 3} & \textbf{Série 4} & \textbf{Série 5} & \textbf{Série 6} \\
%\hline
%\textbf{Tamanho}  & 191672 & 195708   & 142941  & 137257  & 179443  & 177631 \\
%\hline
%\end{tabular}
%\caption{Tamanhos das séries reais}
%\label{tamanho-series-reais}
%\end{center}
%\end{figure}

\subsection{Ambiente}

A implementação de todas as heurísticas foi feita em \verb|C++|, o compilador
utilizado foi \verb|o g++ 4.4.3|, com flag de otimização \verb|-O2|. O computador
utilizado para os testes tem \verb|3GB| de RAM e processador Intel(R) Core(TM)2 Duo CPU, T6600  @ 2.20GHz
de 64 bits.


\subsection{Todos os Pares}

Nessa seção reportamos os experimentos realizados sobre 
o corpus para o problema de \textit{todos os pares}.

Como mencionado, existem duas estruturas principais para esse problema, a f-pair, nossa proposta,
e outra, usada como baseline, que é somente uma RMQ (chamaremos essa estrutura
de RMQStructure). Ambas estruturas dependem de uma implementação de RMQ, porém, como observado,
 existem diferentes algoritmos para o problema de RMQ. Como apresentado na seção \ref{lalal}, na prática, a \verb|RMQ Bucket|
 apresenta os melhores resultados dentre que têm pré-processamento linear.
Contudo, a \verb|RMQ St| apresenta o melhor tempo de resposta as pesquisas, apesar do tempo 
e espaço de pré-processamento de $\theta(n \log n)$.

A heurística f-pair, no nosso ambiente, apresentou consistentemente melhores resultados 
quando baseada na implementação \verb|RMQ Bucket|. Em outra linha, contudo, a RMQStructure apresenta
melhores resultados com o \verb|RMQ St|, desde que o espaço de pré-processamento 
não seja proibitivo (nos testes com séries aleatórias, esse algoritmo
consumia quase toda memória real do sistema para a série de $5$MB). Sendo assim, criamos duas versões
do RMQStructure, uma que se basea na \verb|RMQ Bucket| e 
outra baseada na \verb|RMQ St|, chamamos a primeira versão de \verb|RMQBucket|
e a segunda de \verb|RMQSt| e reportamos os resultados para ambas versões.

\subsubsection{Séries Aleatórias}

Relembrando, este corpus consiste de 5 séries, com tamanhos variando de $1$MB até $5$MB.
Cada série é uma permutação gerada aleatoriamente. Na Figura~\ref{pre-random},
 temos os tempos que cada estrutura levou para ser criada durante o pré-processamento
e o espaço ocupado por cada estrutura. Note que a quantidade de f-pair e o espaço
ocupado pela série são aderentes a análise teórica.

\begin{figure}
\begin{center}
\begin{tabular}{|l|l|l|l|l|l|l|}
\hline
      &    & Série 1 & Série 2 & Série 3 & Série 4 & Série 5 \\
\hline
			& \textbf{Tamanho}  & 1048576 & 2097152  & 3145728  & 4194304  & 5242880 \\
\hline
			& \textbf{Total f-pairs}  & 1049258 & 2096726   & 3146401  & 4193473  & 5247841 \\
\hline
\multirow{2}{*}{\textbf{Fpair}} & Espaço & 109.7 & 228 & 336.8 & 444.5 & 574.9\\
                                & Tempo & 618.336 & 1279.97 & 1976.78 & 2406.85 & 3372.51\\
\hline
\multirow{2}{*}{\textbf{RMQBucket}} & Espaço & 23.2 & 45.1 & 71.8 & 93.5 & 116.8\\
													          & Tempo & 114.921 & 286.451 & 527.042 & 455.153 & 755.968\\
\hline
\multirow{2}{*}{\textbf{RMQSt}} & Espaço & 248.4 & 560.4 & 840 & 1.16GB & 1.46GB \\
											          & Tempo & 112.431 & 290.094 & 528.019 & 465.001 & 759.326\\
\hline
\end{tabular}
\caption{Tempo (milesegundos) e espaço(megabytes, ao menos se especificado diferente) de pré-processamento para cada estrutura.}
\label{pre-random}
\end{center}
\end{figure}



Para esse conjunto fizemos dois tipos de experimentos (cada experimento é um conjunto de consultas).
O primeiro contém consultas com poucas respostas até consultas com número de resposta próximo
ao tamanho da série, já no segundo,  escolhemos consultas $(t, d)$ que retornem poucos pares, essas
consultas são o foco principal desse trabalho, uma vez que estamos interessados
em eventos raros. Nas Figuras~\ref{1-large-random} até ~\ref{5-large-random} estão
os tempos reportados para o primeiro experimento e nas Figuras~\ref{1-small-random} até~\ref{5-small-random}
estão as tabelas para o segundo conjunto, de saídas menores. 
O tempo reportado para uma consulta é calculado através de $5$ (cinco) repetições da mesma
consulta, isto é, o tempo de cada repetição é medido em milisegundos e a mediana dos tempos
das repetições é reportada.

Os parâmetros $(t, d)$ para gerar saídas pequenas foram formados da seguinte forma. A primeira pesquisa
foi feita com $t = 4$ e $d = n - 1000$ ($n$ é tamanho da série) e cada pesquisa subsequente foi formada a partir da anterior somando-se
$200$ ao valor de $t$ e subtraindo $20$ do valor de $d$. Note que para todas as seis séries do conjunto foram usados os mesmos
$t$'s e $d$'s, de forma que essa sequência de geração foi escolhida por retornar valores pequenos em todas as séries.


Para as saídas grandes, os resultados sugerem que para a \verb|RMQStructure| os tempos são sempre melhores usando a implementação
com pré-processando $O(n \log n)$ para RMQ (RMQST), porém, devido ao consumo de mémoria não é possível
usar essa estrutura com a quinta série. Já a estrutura f-pair, para as séries de $1$ a $4$, se mostra melhor
até um certo tamanho de saída depois começa a ter tempo de resposta pior que a \verb|RMQSt|. Para série $5$,
ela é melhor em todas as pesquisas. Em outra linha, para as séries com saídas menores, nossa estrutura é bastante rápida.
Isto é um ponto positivo, pois, sobre o foco eventos raros, essas são as pesquisas mais interessantes.


Os gráficos apresentados mostram o tamanho da saída no eixo X e os tempos no eixo Y. Para cada estratégia,
além dos pontos, é apresentada a reta que melhor aproxima o conjunto de pontos, formalmente, a reta
que minimiza a soma das distâncias euclidianas dos ponto em relação a ela. Note que essa reta parece modelar
bem o comportamento de cada estratégia. Isto acontece pois como apresentado na parte teórica, o tempo de consulta
para a estratégia RmqSt é linear, para estratégia rmqBucket é $n\log n$ e, o tempo esperado da f-pair é $n\log n$
quando a saída é da ordem do tamanho da série. Ou seja, os tempos de consultas são ou uma função linear ou uma próxima. 

\begin{figure}
\begin{center}
\includegraphics{/home/cvalentim/Mestrado/research/data_structures/imp/report/outs/22_nov/random_s1/graphics.eps}
\caption{Série Aleatória 1 - Saídas Grandes}
\end{center}
\end{figure}

\begin{figure}
\begin{center}
\includegraphics{/home/cvalentim/Mestrado/research/data_structures/imp/report/outs/22_nov/random_s2/graphics.eps}
\caption{Série Aleatória 2 - Saídas Grandes}
\end{center}
\end{figure}

\begin{figure}
\begin{center}
\includegraphics{/home/cvalentim/Mestrado/research/data_structures/imp/report/outs/22_nov/random_s3/graphics.eps}
\caption{Série Aleatória 3 - Saídas Grandes}
\end{center}
\end{figure}

\begin{figure}
\begin{center}
\includegraphics{/home/cvalentim/Mestrado/research/data_structures/imp/report/outs/22_nov/random_s4/graphics.eps}
\caption{Série Aleatória 4 - Saídas Grandes}
\end{center}
\end{figure}

\begin{figure}
\begin{center}
\includegraphics{/home/cvalentim/Mestrado/research/data_structures/imp/report/outs/22_nov/random_s5/graphics.eps}
\caption{Série Aleatória 5 - Saídas Grandes}
\end{center}
\end{figure}

% Table 1 && Table 2 - Random with small output
\clearpage
\begin{figure}
\begin{center}
\begin{tabular}{|c|c|c|c|c|c|}
\hline
\textbf{t} & \textbf{d} & \textbf{saída} & \textbf{fpair} & \textbf{rmqBucket} & \textbf{rmqSt}\\
\hline
204 & 3144708 & 42 & 0.07 & 779.33 & 787.57\\
\hline
404 & 3144688 & 79 & 0.14 & 827.05 & 829.64\\
\hline
604 & 3144668 & 121 & 0.22 & 841.87 & 845.9\\
\hline
804 & 3144648 & 143 & 0.4 & 823.6 & 882.85\\
\hline
1004 & 3144628 & 176 & 0.46 & 837.46 & 904.1\\
\hline
1204 & 3144608 & 226 & 1.31 & 828.2 & 886.54\\
\hline
1404 & 3144588 & 280 & 1.26 & 832.26 & 897.97\\
\hline
1604 & 3144568 & 350 & 1.36 & 828.07 & 848.04\\
\hline
1804 & 3144548 & 408 & 1.68 & 959.11 & 841.35\\
\hline
2004 & 3144528 & 466 & 1.9 & 874.13 & 831.68\\
\hline
2204 & 3144508 & 517 & 2.29 & 835.73 & 845.98\\
\hline
2404 & 3144488 & 594 & 2.31 & 832.53 & 841.27\\
\hline
\end{tabular}

\caption{Série Aleatória 1 - Saídas Pequenas}
\label{1-small-random}
\end{center}
\end{figure}

\begin{figure}
\begin{center}
\begin{tabular}{|c|c|c|c|c|c|}
\hline
\textbf{t} & \textbf{d} & \textbf{saída} & \textbf{fpair} & \textbf{rmqBucket} & \textbf{rmqSt}\\
\hline
204 & 3144708 & 42 & 0.07 & 779.33 & 787.57\\
\hline
404 & 3144688 & 79 & 0.14 & 827.05 & 829.64\\
\hline
604 & 3144668 & 121 & 0.22 & 841.87 & 845.9\\
\hline
804 & 3144648 & 143 & 0.4 & 823.6 & 882.85\\
\hline
1004 & 3144628 & 176 & 0.46 & 837.46 & 904.1\\
\hline
1204 & 3144608 & 226 & 1.31 & 828.2 & 886.54\\
\hline
1404 & 3144588 & 280 & 1.26 & 832.26 & 897.97\\
\hline
1604 & 3144568 & 350 & 1.36 & 828.07 & 848.04\\
\hline
1804 & 3144548 & 408 & 1.68 & 959.11 & 841.35\\
\hline
2004 & 3144528 & 466 & 1.9 & 874.13 & 831.68\\
\hline
2204 & 3144508 & 517 & 2.29 & 835.73 & 845.98\\
\hline
2404 & 3144488 & 594 & 2.31 & 832.53 & 841.27\\
\hline
\end{tabular}
\\
\caption{Série Aleatória 2 - Saídas Pequenas}
\label{2-small-random}
\end{center}
\end{figure}


% Table 3 && Table 4 - Random with small output
\begin{figure}
\begin{center}
\begin{tabular}{|c|c|c|c|c|c|}
\hline
\textbf{t} & \textbf{d} & \textbf{saída} & \textbf{fpair} & \textbf{rmqBucket} & \textbf{rmqSt}\\
\hline
204 & 3144708 & 42 & 0.07 & 779.33 & 787.57\\
\hline
404 & 3144688 & 79 & 0.14 & 827.05 & 829.64\\
\hline
604 & 3144668 & 121 & 0.22 & 841.87 & 845.9\\
\hline
804 & 3144648 & 143 & 0.4 & 823.6 & 882.85\\
\hline
1004 & 3144628 & 176 & 0.46 & 837.46 & 904.1\\
\hline
1204 & 3144608 & 226 & 1.31 & 828.2 & 886.54\\
\hline
1404 & 3144588 & 280 & 1.26 & 832.26 & 897.97\\
\hline
1604 & 3144568 & 350 & 1.36 & 828.07 & 848.04\\
\hline
1804 & 3144548 & 408 & 1.68 & 959.11 & 841.35\\
\hline
2004 & 3144528 & 466 & 1.9 & 874.13 & 831.68\\
\hline
2204 & 3144508 & 517 & 2.29 & 835.73 & 845.98\\
\hline
2404 & 3144488 & 594 & 2.31 & 832.53 & 841.27\\
\hline
\end{tabular}

\label{3-small-random}
\caption{Série Aleatória 3 - Saídas Pequenas}
\end{center}
\end{figure}

\begin{figure}
\begin{center}
\begin{tabular}{|c|c|c|c|c|c|}
\hline
\textbf{t} & \textbf{d} & \textbf{saída} & \textbf{fpair} & \textbf{rmqBucket} & \textbf{rmqSt}\\
\hline
204 & 3144708 & 42 & 0.07 & 779.33 & 787.57\\
\hline
404 & 3144688 & 79 & 0.14 & 827.05 & 829.64\\
\hline
604 & 3144668 & 121 & 0.22 & 841.87 & 845.9\\
\hline
804 & 3144648 & 143 & 0.4 & 823.6 & 882.85\\
\hline
1004 & 3144628 & 176 & 0.46 & 837.46 & 904.1\\
\hline
1204 & 3144608 & 226 & 1.31 & 828.2 & 886.54\\
\hline
1404 & 3144588 & 280 & 1.26 & 832.26 & 897.97\\
\hline
1604 & 3144568 & 350 & 1.36 & 828.07 & 848.04\\
\hline
1804 & 3144548 & 408 & 1.68 & 959.11 & 841.35\\
\hline
2004 & 3144528 & 466 & 1.9 & 874.13 & 831.68\\
\hline
2204 & 3144508 & 517 & 2.29 & 835.73 & 845.98\\
\hline
2404 & 3144488 & 594 & 2.31 & 832.53 & 841.27\\
\hline
\end{tabular}
 \\
\label{4-small-random}
\caption{Série Aleatória 4 - Saídas Pequenas}
\end{center}
\end{figure}


% Table 6 - Random with small output
\begin{figure}
\begin{center}
\begin{tabular}{|c|c|c|c|c|c|}
\hline
\textbf{t} & \textbf{d} & \textbf{saída} & \textbf{fpair} & \textbf{rmqBucket} & \textbf{rmqSt}\\
\hline
204 & 3144708 & 42 & 0.07 & 779.33 & 787.57\\
\hline
404 & 3144688 & 79 & 0.14 & 827.05 & 829.64\\
\hline
604 & 3144668 & 121 & 0.22 & 841.87 & 845.9\\
\hline
804 & 3144648 & 143 & 0.4 & 823.6 & 882.85\\
\hline
1004 & 3144628 & 176 & 0.46 & 837.46 & 904.1\\
\hline
1204 & 3144608 & 226 & 1.31 & 828.2 & 886.54\\
\hline
1404 & 3144588 & 280 & 1.26 & 832.26 & 897.97\\
\hline
1604 & 3144568 & 350 & 1.36 & 828.07 & 848.04\\
\hline
1804 & 3144548 & 408 & 1.68 & 959.11 & 841.35\\
\hline
2004 & 3144528 & 466 & 1.9 & 874.13 & 831.68\\
\hline
2204 & 3144508 & 517 & 2.29 & 835.73 & 845.98\\
\hline
2404 & 3144488 & 594 & 2.31 & 832.53 & 841.27\\
\hline
\end{tabular}
\\
\caption{Série Aleatória 5 - Saídas Pequenas}
\label{5-small-random}
\end{center}
\end{figure}




\subsubsection{Séries Reais}

Para as séries reais são reportadas consultas com número de respostas pequeno
em relação ao tamanho da série. Para cada uma das seis séries foram feitas
9 consultas. Os tempos (em milisegundos) que cada estrutura levou para responder
cada consulta é reportado nas Figuras~\ref{1-small-random} até~\ref{5-small-random}
. Note que a estrutura f-pair obtém os menores tempos de resposta,
e que, em algum casos, os tempos são até 2000 vezes menores.

Na Figura~\ref{pre-real} temos o tempo e espaço que cada estrutura levou para pré-processar
as séries. Note que como esses dados são números em ponto-flutuante, cada elemento
da série é representado com $8$ bytes, e não com $4$, como nas séries aleatórias compostas
de números inteiros.

\begin{figure}
\begin{tabular}{|l|l|l|l|l|l|l|l|}
\hline
      &    & Série 1 & Série 2 & Série 3 & Série 4 & Série 5 & Série 6\\
\hline
			& \textbf{Tamanho}  & 191672 & 195708   & 142941  & 137257  & 179443  & 177631 \\
\hline
			& \textbf{Total f-pairs}  & 1386964 & 1305974   & 713019  & 968666  & 853435 & 910915 \\
\hline
\multirow{2}{*}{\textbf{Fpair}} & Espaço & 118.6  & 111.3  & 66.4  & 83.2  & 78.6  & 82.1 \\
                       					& Tempo & 617.135 & 482.167 & 264.001 & 356.84 & 313.852 & 328.672\\
\hline
\multirow{2}{*}{\textbf{RMQBucket}} & Espaço & 7.5  & 7.6  & 5.6  & 5.3  & 7.0 & 6.9 \\
													 					& Tempo & 18.256 & 22.043 & 9.154 & 7.457 & 15.479 & 16.264\\
\hline
\multirow{2}{*}{\textbf{RMQSt}} & Espaço &  58.8 & 59.9 & 43.8 & 42.1 & 54.9 & 54.4\\
											 					& Tempo &  429.415 & 517.824 & 269.665 & 250.911 & 342.497 & 334.736\\
\hline
\end{tabular}
\label{pre-real}
\caption{Tempo (em milesegundos) e espaço (em megabytes) de pré-processamento para cada estrutura sobre
o corpus de séries reais.}
\end{figure}

% Séries reais, tamanho de resposta pequeno.
% Table 1 - Real serie with small output
\begin{figure}
\begin{center}
\begin{tabular}{|c|c|c|c|c|c|}
\hline
\textbf{t} & \textbf{d} & \textbf{saída} & \textbf{fpair} & \textbf{rmqBucket} & \textbf{rmqSt}\\
\hline
204 & 3144708 & 42 & 0.07 & 779.33 & 787.57\\
\hline
404 & 3144688 & 79 & 0.14 & 827.05 & 829.64\\
\hline
604 & 3144668 & 121 & 0.22 & 841.87 & 845.9\\
\hline
804 & 3144648 & 143 & 0.4 & 823.6 & 882.85\\
\hline
1004 & 3144628 & 176 & 0.46 & 837.46 & 904.1\\
\hline
1204 & 3144608 & 226 & 1.31 & 828.2 & 886.54\\
\hline
1404 & 3144588 & 280 & 1.26 & 832.26 & 897.97\\
\hline
1604 & 3144568 & 350 & 1.36 & 828.07 & 848.04\\
\hline
1804 & 3144548 & 408 & 1.68 & 959.11 & 841.35\\
\hline
2004 & 3144528 & 466 & 1.9 & 874.13 & 831.68\\
\hline
2204 & 3144508 & 517 & 2.29 & 835.73 & 845.98\\
\hline
2404 & 3144488 & 594 & 2.31 & 832.53 & 841.27\\
\hline
\end{tabular}

\caption{Série Real 1 - Saídas Pequenas}
\label{1-real}
\end{center}
\end{figure}

% Table 2 - Real serie with small output
\begin{figure}
\begin{center}
\begin{tabular}{|c|c|c|c|c|c|}
\hline
\textbf{t} & \textbf{d} & \textbf{saída} & \textbf{fpair} & \textbf{rmqBucket} & \textbf{rmqSt}\\
\hline
204 & 3144708 & 42 & 0.07 & 779.33 & 787.57\\
\hline
404 & 3144688 & 79 & 0.14 & 827.05 & 829.64\\
\hline
604 & 3144668 & 121 & 0.22 & 841.87 & 845.9\\
\hline
804 & 3144648 & 143 & 0.4 & 823.6 & 882.85\\
\hline
1004 & 3144628 & 176 & 0.46 & 837.46 & 904.1\\
\hline
1204 & 3144608 & 226 & 1.31 & 828.2 & 886.54\\
\hline
1404 & 3144588 & 280 & 1.26 & 832.26 & 897.97\\
\hline
1604 & 3144568 & 350 & 1.36 & 828.07 & 848.04\\
\hline
1804 & 3144548 & 408 & 1.68 & 959.11 & 841.35\\
\hline
2004 & 3144528 & 466 & 1.9 & 874.13 & 831.68\\
\hline
2204 & 3144508 & 517 & 2.29 & 835.73 & 845.98\\
\hline
2404 & 3144488 & 594 & 2.31 & 832.53 & 841.27\\
\hline
\end{tabular}
\\
\caption{Série Real 2 - Saídas Pequenas}
\label{2-real}
\end{center}
\end{figure}

% Table 3 - Real serie with small output
\begin{figure}
\begin{center}
\begin{tabular}{|c|c|c|c|c|c|}
\hline
\textbf{t} & \textbf{d} & \textbf{saída} & \textbf{fpair} & \textbf{rmqBucket} & \textbf{rmqSt}\\
\hline
204 & 3144708 & 42 & 0.07 & 779.33 & 787.57\\
\hline
404 & 3144688 & 79 & 0.14 & 827.05 & 829.64\\
\hline
604 & 3144668 & 121 & 0.22 & 841.87 & 845.9\\
\hline
804 & 3144648 & 143 & 0.4 & 823.6 & 882.85\\
\hline
1004 & 3144628 & 176 & 0.46 & 837.46 & 904.1\\
\hline
1204 & 3144608 & 226 & 1.31 & 828.2 & 886.54\\
\hline
1404 & 3144588 & 280 & 1.26 & 832.26 & 897.97\\
\hline
1604 & 3144568 & 350 & 1.36 & 828.07 & 848.04\\
\hline
1804 & 3144548 & 408 & 1.68 & 959.11 & 841.35\\
\hline
2004 & 3144528 & 466 & 1.9 & 874.13 & 831.68\\
\hline
2204 & 3144508 & 517 & 2.29 & 835.73 & 845.98\\
\hline
2404 & 3144488 & 594 & 2.31 & 832.53 & 841.27\\
\hline
\end{tabular}

\caption{Série Real 3 - Saídas Pequenas}
\label{3-real}
\end{center}
\end{figure}

% Table 4 - Real serie with small output
\begin{figure}
\begin{center}
\begin{tabular}{|c|c|c|c|c|c|}
\hline
\textbf{t} & \textbf{d} & \textbf{saída} & \textbf{fpair} & \textbf{rmqBucket} & \textbf{rmqSt}\\
\hline
204 & 3144708 & 42 & 0.07 & 779.33 & 787.57\\
\hline
404 & 3144688 & 79 & 0.14 & 827.05 & 829.64\\
\hline
604 & 3144668 & 121 & 0.22 & 841.87 & 845.9\\
\hline
804 & 3144648 & 143 & 0.4 & 823.6 & 882.85\\
\hline
1004 & 3144628 & 176 & 0.46 & 837.46 & 904.1\\
\hline
1204 & 3144608 & 226 & 1.31 & 828.2 & 886.54\\
\hline
1404 & 3144588 & 280 & 1.26 & 832.26 & 897.97\\
\hline
1604 & 3144568 & 350 & 1.36 & 828.07 & 848.04\\
\hline
1804 & 3144548 & 408 & 1.68 & 959.11 & 841.35\\
\hline
2004 & 3144528 & 466 & 1.9 & 874.13 & 831.68\\
\hline
2204 & 3144508 & 517 & 2.29 & 835.73 & 845.98\\
\hline
2404 & 3144488 & 594 & 2.31 & 832.53 & 841.27\\
\hline
\end{tabular}
\\
\caption{Série Real 4 - Saídas Pequenas}
\label{3-real}
\end{center}
\end{figure}


% Table 5 - Real serie with small output
\begin{figure}
\begin{center}
\begin{tabular}{|c|c|c|c|c|c|}
\hline
\textbf{t} & \textbf{d} & \textbf{saída} & \textbf{fpair} & \textbf{rmqBucket} & \textbf{rmqSt}\\
\hline
204 & 3144708 & 42 & 0.07 & 779.33 & 787.57\\
\hline
404 & 3144688 & 79 & 0.14 & 827.05 & 829.64\\
\hline
604 & 3144668 & 121 & 0.22 & 841.87 & 845.9\\
\hline
804 & 3144648 & 143 & 0.4 & 823.6 & 882.85\\
\hline
1004 & 3144628 & 176 & 0.46 & 837.46 & 904.1\\
\hline
1204 & 3144608 & 226 & 1.31 & 828.2 & 886.54\\
\hline
1404 & 3144588 & 280 & 1.26 & 832.26 & 897.97\\
\hline
1604 & 3144568 & 350 & 1.36 & 828.07 & 848.04\\
\hline
1804 & 3144548 & 408 & 1.68 & 959.11 & 841.35\\
\hline
2004 & 3144528 & 466 & 1.9 & 874.13 & 831.68\\
\hline
2204 & 3144508 & 517 & 2.29 & 835.73 & 845.98\\
\hline
2404 & 3144488 & 594 & 2.31 & 832.53 & 841.27\\
\hline
\end{tabular}

\caption{Série Real 5 - Saídas Pequenas}
\end{center}
\end{figure}

% Table 6 - Real serie with small output
\begin{figure}
\begin{center}
\begin{tabular}{|c|c|c|c|c|c|}
\hline
\textbf{t} & \textbf{d} & \textbf{saída} & \textbf{fpair} & \textbf{rmqBucket} & \textbf{rmqSt}\\
\hline
204 & 3144708 & 42 & 0.07 & 779.33 & 787.57\\
\hline
404 & 3144688 & 79 & 0.14 & 827.05 & 829.64\\
\hline
604 & 3144668 & 121 & 0.22 & 841.87 & 845.9\\
\hline
804 & 3144648 & 143 & 0.4 & 823.6 & 882.85\\
\hline
1004 & 3144628 & 176 & 0.46 & 837.46 & 904.1\\
\hline
1204 & 3144608 & 226 & 1.31 & 828.2 & 886.54\\
\hline
1404 & 3144588 & 280 & 1.26 & 832.26 & 897.97\\
\hline
1604 & 3144568 & 350 & 1.36 & 828.07 & 848.04\\
\hline
1804 & 3144548 & 408 & 1.68 & 959.11 & 841.35\\
\hline
2004 & 3144528 & 466 & 1.9 & 874.13 & 831.68\\
\hline
2204 & 3144508 & 517 & 2.29 & 835.73 & 845.98\\
\hline
2404 & 3144488 & 594 & 2.31 & 832.53 & 841.27\\
\hline
\end{tabular}
\\
\caption{Série Real 6 - Saídas Pequenas}
\end{center}
\end{figure}
\section{Conclusões}

\section{Apêndice A}

\subsection{Séries Aleatórias com saídas grandes}
\clearpage
%Table 1 && Table 2 - Random with large output
\begin{figure}
\begin{center}
\begin{tabular}{|c|c|c|c|c|c|}
\hline
\textbf{t} & \textbf{d} & \textbf{saída} & \textbf{fpair} & \textbf{rmqBucket} & \textbf{rmqSt}\\
\hline
204 & 3144708 & 42 & 0.07 & 779.33 & 787.57\\
\hline
404 & 3144688 & 79 & 0.14 & 827.05 & 829.64\\
\hline
604 & 3144668 & 121 & 0.22 & 841.87 & 845.9\\
\hline
804 & 3144648 & 143 & 0.4 & 823.6 & 882.85\\
\hline
1004 & 3144628 & 176 & 0.46 & 837.46 & 904.1\\
\hline
1204 & 3144608 & 226 & 1.31 & 828.2 & 886.54\\
\hline
1404 & 3144588 & 280 & 1.26 & 832.26 & 897.97\\
\hline
1604 & 3144568 & 350 & 1.36 & 828.07 & 848.04\\
\hline
1804 & 3144548 & 408 & 1.68 & 959.11 & 841.35\\
\hline
2004 & 3144528 & 466 & 1.9 & 874.13 & 831.68\\
\hline
2204 & 3144508 & 517 & 2.29 & 835.73 & 845.98\\
\hline
2404 & 3144488 & 594 & 2.31 & 832.53 & 841.27\\
\hline
\end{tabular}

\caption{Série Aleatória 1 - Saídas Grandes}
\label{1-large-random}
\end{center}
\end{figure}

\begin{figure}
\begin{center}
\begin{tabular}{|c|c|c|c|c|c|}
\hline
\textbf{t} & \textbf{d} & \textbf{saída} & \textbf{fpair} & \textbf{rmqBucket} & \textbf{rmqSt}\\
\hline
204 & 3144708 & 42 & 0.07 & 779.33 & 787.57\\
\hline
404 & 3144688 & 79 & 0.14 & 827.05 & 829.64\\
\hline
604 & 3144668 & 121 & 0.22 & 841.87 & 845.9\\
\hline
804 & 3144648 & 143 & 0.4 & 823.6 & 882.85\\
\hline
1004 & 3144628 & 176 & 0.46 & 837.46 & 904.1\\
\hline
1204 & 3144608 & 226 & 1.31 & 828.2 & 886.54\\
\hline
1404 & 3144588 & 280 & 1.26 & 832.26 & 897.97\\
\hline
1604 & 3144568 & 350 & 1.36 & 828.07 & 848.04\\
\hline
1804 & 3144548 & 408 & 1.68 & 959.11 & 841.35\\
\hline
2004 & 3144528 & 466 & 1.9 & 874.13 & 831.68\\
\hline
2204 & 3144508 & 517 & 2.29 & 835.73 & 845.98\\
\hline
2404 & 3144488 & 594 & 2.31 & 832.53 & 841.27\\
\hline
\end{tabular}
\\
\caption{Série Aleatória 2 - Saídas Grandes}
\label{2-large-random}
\end{center}
\end{figure}


%Table 3 && Table 4 - Random with large output

\begin{figure}
\begin{center}
\begin{tabular}{|c|c|c|c|c|c|}
\hline
\textbf{t} & \textbf{d} & \textbf{saída} & \textbf{fpair} & \textbf{rmqBucket} & \textbf{rmqSt}\\
\hline
204 & 3144708 & 42 & 0.07 & 779.33 & 787.57\\
\hline
404 & 3144688 & 79 & 0.14 & 827.05 & 829.64\\
\hline
604 & 3144668 & 121 & 0.22 & 841.87 & 845.9\\
\hline
804 & 3144648 & 143 & 0.4 & 823.6 & 882.85\\
\hline
1004 & 3144628 & 176 & 0.46 & 837.46 & 904.1\\
\hline
1204 & 3144608 & 226 & 1.31 & 828.2 & 886.54\\
\hline
1404 & 3144588 & 280 & 1.26 & 832.26 & 897.97\\
\hline
1604 & 3144568 & 350 & 1.36 & 828.07 & 848.04\\
\hline
1804 & 3144548 & 408 & 1.68 & 959.11 & 841.35\\
\hline
2004 & 3144528 & 466 & 1.9 & 874.13 & 831.68\\
\hline
2204 & 3144508 & 517 & 2.29 & 835.73 & 845.98\\
\hline
2404 & 3144488 & 594 & 2.31 & 832.53 & 841.27\\
\hline
\end{tabular}

\caption{Série Aleatória 3 - Saídas Grandes}
\label{3-large-random}
\end{center}

\end{figure}

\begin{figure}
\begin{center}
\begin{tabular}{|c|c|c|c|c|c|}
\hline
\textbf{t} & \textbf{d} & \textbf{saída} & \textbf{fpair} & \textbf{rmqBucket} & \textbf{rmqSt}\\
\hline
204 & 3144708 & 42 & 0.07 & 779.33 & 787.57\\
\hline
404 & 3144688 & 79 & 0.14 & 827.05 & 829.64\\
\hline
604 & 3144668 & 121 & 0.22 & 841.87 & 845.9\\
\hline
804 & 3144648 & 143 & 0.4 & 823.6 & 882.85\\
\hline
1004 & 3144628 & 176 & 0.46 & 837.46 & 904.1\\
\hline
1204 & 3144608 & 226 & 1.31 & 828.2 & 886.54\\
\hline
1404 & 3144588 & 280 & 1.26 & 832.26 & 897.97\\
\hline
1604 & 3144568 & 350 & 1.36 & 828.07 & 848.04\\
\hline
1804 & 3144548 & 408 & 1.68 & 959.11 & 841.35\\
\hline
2004 & 3144528 & 466 & 1.9 & 874.13 & 831.68\\
\hline
2204 & 3144508 & 517 & 2.29 & 835.73 & 845.98\\
\hline
2404 & 3144488 & 594 & 2.31 & 832.53 & 841.27\\
\hline
\end{tabular}
 \\
\caption{Série Aleatória 4 - Saídas Grandes}
\label{4-large-random}
\end{center}
\end{figure}

% Table 5 - Random with large output
\begin{figure}
\begin{center}
\begin{tabular}{|c|c|c|c|c|c|}
\hline
\textbf{t} & \textbf{d} & \textbf{saída} & \textbf{fpair} & \textbf{rmqBucket} & \textbf{rmqSt}\\
\hline
204 & 3144708 & 42 & 0.07 & 779.33 & 787.57\\
\hline
404 & 3144688 & 79 & 0.14 & 827.05 & 829.64\\
\hline
604 & 3144668 & 121 & 0.22 & 841.87 & 845.9\\
\hline
804 & 3144648 & 143 & 0.4 & 823.6 & 882.85\\
\hline
1004 & 3144628 & 176 & 0.46 & 837.46 & 904.1\\
\hline
1204 & 3144608 & 226 & 1.31 & 828.2 & 886.54\\
\hline
1404 & 3144588 & 280 & 1.26 & 832.26 & 897.97\\
\hline
1604 & 3144568 & 350 & 1.36 & 828.07 & 848.04\\
\hline
1804 & 3144548 & 408 & 1.68 & 959.11 & 841.35\\
\hline
2004 & 3144528 & 466 & 1.9 & 874.13 & 831.68\\
\hline
2204 & 3144508 & 517 & 2.29 & 835.73 & 845.98\\
\hline
2404 & 3144488 & 594 & 2.31 & 832.53 & 841.27\\
\hline
\end{tabular}
\\
\caption{Série Aleatória 5 - Saídas Grandes}
\label{5-large-random}
\end{center}
\end{figure}

\clearpage

\begin{figure}
\begin{center}
\includegraphics{/home/cvalentim/Mestrado/research/data_structures/imp/report/outs/12_dec/random_s1/graphics.eps}
\caption{Série Aleatória 1 - Saídas Pequenas}
\end{center}
\end{figure}

\begin{figure}
\begin{center}
\includegraphics{/home/cvalentim/Mestrado/research/data_structures/imp/report/outs/12_dec/random_s2/graphics.eps}
\caption{Série Aleatória 2 - Saídas Pequenas}
\end{center}
\end{figure}

\begin{figure}
\begin{center}
\includegraphics{/home/cvalentim/Mestrado/research/data_structures/imp/report/outs/12_dec/random_s3/graphics.eps}
\caption{Série Aleatória 3 - Saídas Pequenas}
\end{center}
\end{figure}

\begin{figure}
\begin{center}
\includegraphics{/home/cvalentim/Mestrado/research/data_structures/imp/report/outs/12_dec/random_s4/graphics.eps}
\caption{Série Aleatória 4 - Saídas Pequenas}
\end{center}
\end{figure}

\begin{figure}
\begin{center}
\includegraphics{/home/cvalentim/Mestrado/research/data_structures/imp/report/outs/12_dec/random_s5/graphics.eps}
\caption{Série Aleatória 5 - Saídas Pequenas}
\end{center}
\end{figure}


\bibliographystyle{plain}

%\bibliography{paper}

\end{document}
