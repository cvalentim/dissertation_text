 \documentclass[12pt,letterpaper]{article}
%\usepackage{ltexpprt}     

\usepackage{amsmath, amsthm, amssymb}

\usepackage[dvips]{graphicx}
%\usepackage{epsf}
\renewcommand{\baselinestretch}{2}
   \usepackage{latexsym}
   \setlength{\oddsidemargin}{-0.2in}
 \setlength{\textwidth}{6.8in}
  \setlength{\topmargin}{-0.7in}
  \setlength{\textheight}{9.4in}
  \begin{document}


\newtheorem{remark}{Remark}
\newtheorem{lemma}{Lemma}
\newtheorem{property}{Property}

\newtheorem{theorem}{Theorem}
\newtheorem{corollary}{Corollary}
\newtheorem{fact}{Fact}
\newtheorem{definition}{Definition}
\newtheorem{proposition}{Proposition}
\newcommand{\DT}{ $D$ }

\title{A data structure for managing financial time series}

\medskip

\normalsize

 \author{Eduardo S. Laber$^1$ \and David Sotelo \and Caio Valentim \\
 laber@inf.puc-rio.br dsilva@inf.puc-rio.br kakaio9@gmail.com  \\
1- Departamento de Inform\' atica, PUC-Rio }


\date{}
%\institute{ Departamento de Inform\' atica, PUC-Rio, Brazil.
% \email{$\{$laber,milidiu,artur$\}$@inf.puc-rio.br}}

\maketitle

\date{}


% If we apply the test $t_i$ on the object  $o_j$, we either have
%$t_i(o_j)=false$ 
% $o_i$ is applied  

\begin{abstract}
\end{abstract}

\bigskip

 {\bf Keywords: } Data Structures, Financial Time Series, Range Queries.
 

\bigskip

 \section{Problem Definition}



Given a  sequence
$A=(a_1,\ldots,a_n)$,
we consider
the 
problem of designing a data structure to efficiently support
the following queries


1.  UP-$(t,d)$  queries. This
query  returns all
pairs $(i,j)$ such that: (i) $i<j$;
(ii) $a_j-a_i \geq d $ and 
(iii) $j-i \leq t$.  
The data structure shall also support
LOW-$(t,d)$ queries, where 
the condition  $a_j-a_i \geq d$ is
replaced with $a_j-a_i \leq - d$.


2. The cardinality version of the UP-$(t,d)$  queries. This
query  returns the number of pairs
 $(i,j)$ such that: (i) $i<j$;
(ii) $a_j-a_i \geq d $ and 
(iii) $j-i \leq t$.  The same
for LOW-$(t,d)$ queries.


3. Include absolute value queries

4. UP-$(t,d,first)$ queries. It 
shall return the set 
$\{ (a,b) | (a,b)$ is a solution of the UP-$(t,d)$
query and for each $c<b$, $(a,c)$ is not a solution of the UP-$(t,d)$ query $\}$


5. Given a fixed interval, 
$[ini,last]$ and $d$,
find all pairs $(i,j)$ such
that $ini \leq i < j \leq last$ and
$a_j-a_i \geq d$. This should be a classical problem solved in 
comp geometry.

6. Approximation queries


\section{The Real FS-Data Structure}

% Inicialmentre explicar sobre Range Minimum Queries e Cartesian Trees,
% dizendo que nossa estrutura cai nesse subproblema.
% Depois falar sobre a estrutura de dados em si:
%
% - falar que mantemos uma lista L com todos os pares satisfazedo $ap_<= ...$
% e $a_q >= ...$ e que essa lista eh ordenada crescentemente por $q-p$.
% Temos que dizer que eh possivel construir essa lista ordenada em $O(m + n)$,
% onde  $m$ eh a quantidade de pares satisfazendo a restricao, e $n$ eh a quantidade
% de elementos da serie temporal, utilizando counting ou radix sort, uma vez que as diferencas
% $q-p$ estao entre 1 e $n$.
%
%- depois, mantemos um ponteiro para o ultimo elemento da lista
%
%
%
%
%
In this section we introduce a data structure that supports answering UP-$(t,d)$ queries
over a financial time series $A$ with $O(|A|)$ expected preprocessing time , $O(|A|)$ expected space
and $O(k + 1)$ worst-case query time, where $k$ is the number of elements returned by the UP-$(t,d)$ query.
The expected processing time and space values come from the fact that we are going to analyze
our data structures from an average case perspective.

Three auxiliar data structures are kept for answering  UP-$(t,d)$ queries, being constructed at preprocessing time.
Ths first one is a list $L$ of interesting pairs of elements of $A$ that will guide our future search procedure for answering queries.
The second one is a simple hash table $H$ that stores pointers for some elements in $L$.
The third type of structure are Cartesian Trees, $T_{L}$  and $T_{A}$, used for answering range queries over $L$ and $A$,
respectively. We proceed now by describing each one of these structures.

%and $n$ is the number of elements in the time series.

%First, we construct structures to 
%handle both range minimum and maximum queries over $A$

Our first and most important data structure consists of a list $L$ that  stores all pairs $(p,q)$, with $p<q$,  such that
$ a_p < \min \{a_{p+1},\ldots,a_{q}\}$ and
$ a_q > \max \{a_{p},\ldots,a_{q-1}\}$.
%The time gap of a pair $(p,q)$ is $q-p$.
%========================================================================================================================
A first point to be observed is that $L$ can be generated in $O(|L| + |A|)$.
It comes from the following property. Let $L_{p} \subseteq L$ denote the subset of pairs in $L$ with first
index $p$. Furthermore, let $n$ be the minimum positive value such that $a(p + n) \leq a(p)$.
Then, $L_{p} = (p, p + 1) \cup L_{p + 1} \cup \ldots \cup (p, p + n) \cup L_{p + n}$.
Finally, $L = \bigcup \limits_{p=1}^{|A|-1} L_{p}$.



%========================================================================================================================

%========================================================================================================================
After constructing $L$, we keep it sorted  by non-decreasing order of time gaps,
where the {\it time gap} of a pair $(p,q) \in L$ is defined as $q-p$.
%$L$ can also be sorted in $O(m + n)$, where $m$ is the number of pairs $(p,q)$
%stored on $L$ and $n$ is the number of elements in the original time series $A$.
It can be done in $O(|L| + |A|)$ time, by using a linear time sorting algorithm
(like radix or counting sort), since the time gaps are between $1$ and $|A|-1$.
%========================================================================================================================


%\begin{example}
%\end{example}

Once $L$ is buildt, we proceed by creating a second auxiliar structure.
It consists of a hash table $H : \{1,\ldots,|A|-1\} \mapsto L$ that stores, for every $t \in \{1,\ldots,|A|-1\}$, the last position
of $L$ associated with a pair of time gap at most $t$. This hash table can also be constructed in $O(|L| + |A|)$ time by
a simple scanning of the entire sorted list $L$.


%it will be useful to know the last position of $L$
%that is associated with a pair of time gap $t$.
%For that, we construct a vector $V$ that stores for every $t \in \{1,\ldots,n\}$,
%the last position of $L$ associated with  a pair of time gap at most $t$.

Finally, our third structure is a Cartesian Tree, a data structure used for 
answering range minimum (maximum) queries. The {\tt Range Minimum (Maximum) Query Problem (RMQ)}
consists of proprocessing  an array $A[1,n]$ in order to answer the following queries:
given two indices $i$ and $j$, $1 \leq i \leq j \leq n$, return the index of the
smallest (biggest) element in the subarray $A[i,j]$.
Cartesian trees \cite{cartesian80} are well-known for solving the RMQ problem with $O(1)$ query time,
$O(n)$ preprocessing time and $O(n)$ space complexity.

Two differents Cartesian Trees will be kept, since we will be interested on answering two distinct types of range minimum
(maximum) queries, one for the original time series $A$ and other for the list of pairs $L$.
In the first type of RMQ, related to the original time series $A$, we ask for the index of the minimum (maximum) value in the subarray
$A[i,j]$, where the value of the $k$th element in $A$ is represented by $a_k$.
In the second type of RMQ, related to the list of pairs $L$,
we ask for the index of the minimum (maximum) value in the subarray
$L[i,j]$, where the value of the $k$th element $(p,q)$ is represented by $a_p - a_q$.



%over two arrays, one representing the original time series $A$ and
%another storing the pairs  $(p,q) \in L$ 


%The {\em deviation} of a pair $(p,q)$ is defined as $a_p-a_q$.
%To efficiently handle our time series queries  we shall construct a RMQ structure associated with $L$.
%Given an interval $(i,j)$ in $L$, this structures is able to retrieve the pair with largest deviation 
%in this interval in $O(1)$ time.

\pagebreak

\section{Appendix}

\begin{proposition}
Let $S$ be a list of special pairs of a time series given by a random permutation from a set 
of $n$ distinct elements. Then, the expected size of $S$ is $n - H_n$, where $H_n$ is the $n$-th harmonic number.
\end{proposition}
\begin{proof}
Let $E[X]$ represent the expected size of the list $S$ of special pairs.
Furthermore, let $X_{i,j}$ denote a random indicator variable
that stores $1$ if the pair of time indexes $(i,j)$ belongs to $S$ and $0$ otherwise.
From the previous definitions, $E[X] = E[\sum\limits_{i=1}^{n-1} \sum\limits_{j=i+1}^{n}X_{i,j}]$.

By the linearity of expectation, it follows that:

$E[\sum\limits_{i=1}^{n-1} \sum\limits_{j=i+1}^{n} X_{i,j}] = \sum\limits_{i=1}^{n-1} \sum\limits_{j=i+1}^{n} E[X_{i,j}]$.

Since that $E[X_{i,j}] = \frac{1}{(j-i+1)(j-i)} = \frac{1}{j-i} - \frac{1}{j-i+1}$, we have:

$E[X] = \sum\limits_{i=1}^{n-1} \sum\limits_{j=i+1}^{n}
(\frac{1}{j-i} - \frac{1}{j-i+1})
= \sum\limits_{i=1}^{n-1} (1 - \frac{1}{n-i+1})
= n - \sum\limits_{i=1}^{n} \frac{1}{i}  =  n - H_n$.
\end{proof}


\noindent It will be demonstrated now that, with high probability bounds,
the size of $S$ concentrates around its mean value. More precisely, we will estimate now the tail distribution of the random variable $X$,
representing the expected size of a list of special pairs $S$ of a
a time series given by a random permutation from a set 
of $n$ distinct elements.



\begin{proposition}
%Let $c > 0$ be a real number, then $Pr[X \geq c.n] \leq \frac{\log n}{c^2 n}$
Let $c > 0$ be a real number, then $Pr\{X \geq c .n \} \leq \frac{c' \log n}{c^2 n}$,
where $c'$ is a fixed positive constant.
\end{proposition}

\begin{proof}

\noindent Chebyshev's inequality states that if $X$ is a random variable with expectation $E[X]$ and variance $Var[X]$,
then $Pr\{ |X - E[X]| \geq c \} \leq \frac{Var[X]}{c^2}$, for any positive constant $c$ given.
Considering that, by definition,  $Var[X] = E[X^2] - E[X]^2$, and multiplying $c$ by $E[x]$ in the previous inequality, we have
$Pr\{ |X - E[X]| \geq c E[X] \} \leq \frac{E[X^2] - E[X]^2}{c^2 E[X]^2}$.

In order to make use of Chebyshev's inequality it is necessary to determine $E[X]^2$ and  $E[X^2]$.

From Proposition 1, it follows that $E[X]^2 = (n - H_{n})^2$.

By definition, $E[X^2] = E [( \sum\limits_{i=1}^{n-1} \sum\limits_{j=i+1}^{n} X_{i,j})^2]$.
Hence, by the linearity of expectation, $E[X^2] = ( 2 \sum\limits_{i=1}^{n-1} \sum\limits_{j=i+1}^{n}
\sum\limits_{k=i+1}^{n-1} \sum\limits_{l=k+1}^{n} E[X_{i,j} X_{k,l}] ) +
( \sum\limits_{i=1}^{n-1} \sum\limits_{j=i+1}^{n} E[(X_{i,j})^2] )$.


We proceed now by calculating $E[X^2]$. It can be splitted into six disjoint cases.

In each one of the following cases we assume that the product of every pair  
of random variables $X_{i,j}$ and $X_{k,l}$  is represented by the intersection
of two closed intervals $[i,j]$ and $[k,l]$, where $[i,j] = \{p | i \leq p \leq j \}$
and $[k,l] = \{q | k \leq q \leq l \}$.

Furthermore, by means of an abuse of notation, we assume in each one of the following cases that
the expression $E[X_{i,j} X_{k,l}]$ denotes $\sum\limits_{i}\sum\limits_{j >i}\sum\limits_{k \geq i}\sum\limits_{l > k} E[X_{i,j} X_{k,l}]$, where $[i,j]$ and $[k,l]$ are pairs of intervals belonging to the related case.


{\bf Case 1:} Intervals  $[i, j]$ and $[k, l]$ are identical.

$E[X_{i,j} X_{k,l}] = E[X_{i,j} X_{i,j}] = E[X_{i,j}] = \sum\limits_{i=1}^{n-1} \sum\limits_{j=i+1}^{n}
\frac{1}{(j-i+1)(j-i)} = n - H_n$

\vspace{0.5cm}


{\bf Case 2a:} Intervals  $[i, j]$ and $[k, l]$ are disjoint.

$E[X_{i,j} X_{k,l}] = \sum\limits_{i=1}^{n-3} \sum\limits_{j=i+1}^{n-2}
\sum\limits_{k=j+1}^{n-1} \sum\limits_{l=k+1}^{n} \frac{1}{(j-i+1)(j-i)(l-k+1)(l-k)} = $

$\sum\limits_{i=1}^{n-3} \sum\limits_{j=i+1}^{n-2}  \left[ \frac{1}{(j-i+1)(j-i)}
\left( \sum\limits_{k=j+1}^{n-1} \sum\limits_{l=k+1}^{n} \frac{1}{(l-k+1)(l-k)} \right) \right] =  $

$\sum\limits_{i=1}^{n-3} \sum\limits_{j=i+1}^{n-2} \frac{(n - j) - H_{(n - j)}} {(j-i+1)(j-i)} 
= \alpha(n) - \beta(n) - \gamma(n)$, where

$\alpha(n) = \sum\limits_{i=1}^{n-3} \sum\limits_{j=i+1}^{n-2} \frac{n} {(j-i+1)(j-i)}$,
$\beta(n) = \sum\limits_{i=1}^{n-3} \sum\limits_{j=i+1}^{n-2} \frac{j} {(j-i+1)(j-i)}$ and 
$\gamma(n) =  \sum\limits_{i=1}^{n-3} \sum\limits_{j=i+1}^{n-2} \frac{H_{(n - j)}} {(j-i+1)(j-i)}$
%\sum\limits_{i=1}^{n-3} \sum\limits_{j=i+1}^{n-2} \frac{n-j} {(j-i+1)(j-i)}

Let us first calculate $\alpha(n)$:

$\alpha(n) = \sum\limits_{i=1}^{n-3} \sum\limits_{j=i+1}^{n-2} \frac{n} {(j-i+1)(j-i)}
= n^2 - 2n - H_{(n-2)}$

From $\beta(n)$ definiton, it follows that:

$\beta(n) = \sum\limits_{i=1}^{n-3} \sum\limits_{j=i+1}^{n-2} \frac{j} {(j-i+1)(j-i)}
= \sum\limits_{i=1}^{n-3} \left( \sum\limits_{j=i+1}^{n-2} \frac{j}{(j-i)} - \frac{j} {(j-i+1)} \right)$

$\beta(n) = \sum\limits_{i=1}^{n-3} \left[ \left( \frac{i+1}{1} +  \frac{i+2}{2} + \ldots + \frac{n-2}{n-i-2} \right)
- \left( \frac{i+1}{2} +  \frac{i+2}{3} + \ldots + \frac{n-2}{n-i-1} \right) \right]$

$\beta(n) = \sum\limits_{i=1}^{n-3} \left[
\frac{i+1}{1}  + \left( \frac{1}{2} +  \frac{1}{3} + \ldots + \frac{1}{n-i-2} \right)
- \frac{n-2}{n-i-1} \right]
= \sum\limits_{i=1}^{n-3} \left( i + H_{(n-2)} - \frac{n-i-2}{n-i-1} \right)$

$\beta(n) = \frac{(n-3)(n-2)}{2} + \sum\limits_{i=1}^{n-3} H_{i} - \sum\limits_{i=2}^{n-2} \frac{n-2}{i}$

$\beta(n) = \frac{n^2}{2} - \Theta(n \log n)$.

Similarly, a tight bound can be determined for $\gamma(n)$:

$\gamma(n) =  \sum\limits_{i=1}^{n-3} \sum\limits_{j=i+1}^{n-2} \frac{H_{(n - j)}} {(j-i+1)(j-i)}
\leq  \sum\limits_{i=1}^{n-3} \sum\limits_{j=i+1}^{n-2} \frac{H_{n}} {(j-i+1)(j-i)}
\leq  n . H(n) = O(n \log n) $

$\gamma(n) =  \sum\limits_{i=1}^{n-3} \sum\limits_{j=i+1}^{n-2} \frac{H_{(n - j)}} {(j-i+1)(j-i)}
\geq  \sum\limits_{i=1}^{\frac{n-2}{2} - 1} \sum\limits_{j=i+1}^{\frac{n-2}{2}} \frac{H_{(n/2)}} {(j-i+1)(j-i)}
\geq  H(n/2) . \left[\frac{n-2}{2} - H_{(n-2)/2} \right] = \Omega(n \log n) $

Therefore, $\gamma(n) = \Theta(n \log n)$.


Returning to the original equation:

$\sum\limits_{i=1}^{n-3} \sum\limits_{j=i+1}^{n-2} \frac{(n - j) - H_{(n - j)}} {(j-i+1)(j-i)} 
= \alpha(n) - \beta(n) - \gamma(n)
\leq  \frac{n^2}{2} + \Theta(n \log n)$

\vspace{0.5cm}

{\bf Case 2b:} Interval $[k, l]$ is a subinterval of $[i, j]$.

$E[X_{i,j} X_{k,l}] = \sum\limits_{i=1}^{n-3} \sum\limits_{k=i+1}^{n-2}
\sum\limits_{l=k+1}^{n-1} \sum\limits_{j=l+1}^{n} \frac{1}{(j-i+1)(j-i)(l-k+1)(l-k)}
\leq  \sum\limits_{i=1}^{n-3} \sum\limits_{k=i+1}^{n-2}
\sum\limits_{l=k+1}^{n-1} \sum\limits_{j=l+1}^{n} \frac{1}{(l-k+1)(l-k)(l-k+1)(l-k)}$

$E[X_{i,j} X_{k,l}] \leq\sum\limits_{i=1}^{n-3} \sum\limits_{k=i+1}^{n-2}
\sum\limits_{l=k+1}^{n-1} \sum\limits_{j=l+1}^{n} \frac{1}{(l-k)^4}
\leq n \sum\limits_{i=1}^{n-3} \sum\limits_{k=i+1}^{n-2}
\sum\limits_{l=k+1}^{n-1}  \frac{1}{(l-k)^4} $

$E[X_{i,j} X_{k,l}] \leq n \sum\limits_{i=1}^{n-3} \sum\limits_{k=i+1}^{n-2}
\left( \displaystyle \int\limits_{k}^{n-2} \! \frac{1}{(l-k)^4} \, \mathrm{d}l \right)
\leq 3n \sum\limits_{i=1}^{n-3} \sum\limits_{k=i+1}^{n-2} \frac{1}{k^3} $

$E[X_{i,j} X_{k,l}]  \leq 3n \sum\limits_{i=1}^{n-3}
\left( \displaystyle \int\limits_{i}^{n-3} \! \frac{1}{k^3} \, \mathrm{d}k \right)
\leq 6n \sum\limits_{i=1}^{n-3} \frac{1}{i^2}
 \leq 6n \left( 1 + \displaystyle \int\limits_{1}^{n-4} \! \frac{1}{i^2} \, \mathrm{d}i \right)
 \leq 18n$

\vspace{0.5cm}

{\bf Case 3:} Intersection between intervals $[i, j]$ and $[k, l]$ with no shared extremes.

$E[X_{i,j} X_{k,l}] = \sum\limits_{i=1}^{n-3} \sum\limits_{k=i+1}^{n-2}
\sum\limits_{j=k+1}^{n-1} \sum\limits_{l=j+1}^{n} \frac{1}{(l-i+1)(l-i)(j-i)(l-k-1)}$

This sum can be writen as:

$E[X_{i,j} X_{k,l}] = \sum\limits_{i=1}^{n-3} \sum\limits_{j=i+2}^{n-1}
\sum\limits_{k=i+1}^{j-1} \sum\limits_{l=j+1}^{n} \frac{1}{(l-i+1)(l-i)(j-i)(l-k-1)}$


Since that $j < l$ and $ l - k < j - i$, we have:

$E[X_{i,j} X_{k,l}] < \sum\limits_{i=1}^{n-3} \sum\limits_{j=i+2}^{n-1}
\sum\limits_{k=i+1}^{j-1} \sum\limits_{l=j+1}^{n} \frac{1}{(j-i+1)(j-i)(l-k)(l-k-1)}$

$E[X_{i,j} X_{k,l}] < \sum\limits_{i=1}^{n-3} \sum\limits_{j=i+2}^{n-1}
\left[ \frac{1}{(j-i+1)(j-i)}  \left( \sum\limits_{k=i+1}^{j-1} \sum\limits_{l=j+1}^{n} \frac{1}{(l-k)(l-k-1)} \right) \right]$

$E[X_{i,j} X_{k,l}] < \sum\limits_{i=1}^{n-3} \sum\limits_{j=i+2}^{n-1}
\left[ \frac{1}{(j-i+1)(j-i)}  \sum\limits_{k=i+1}^{j-1} \sum\limits_{l=j+1}^{n}  \left(  \frac{1}{l-k-1} - \frac{1}{l-k}  \right) \right]$

$E[X_{i,j} X_{k,l}] < \sum\limits_{i=1}^{n-3} \sum\limits_{j=i+2}^{n-1}
\left[ \frac{1}{(j-i+1)(j-i)}  \sum\limits_{k=i+1}^{j-1}  \left(  \frac{1}{j-k} - \frac{1}{n-k}  \right) \right]$

$E[X_{i,j} X_{k,l}] < \sum\limits_{i=1}^{n-3} \sum\limits_{j=i+2}^{n-1}
\frac{H_{n}}{(j-i+1)(j-i)} $

Therefore, $E[X_{i,j} X_{k,l}] = O(n \log n)$.

\vspace{0.5cm}

{\bf Case 4:} Intersection between intervals $[i, j]$ and $[k, l]$ with $i = k$.

$E[X_{i,j} X_{k,l}] = \sum\limits_{i=1}^{n-2}
\sum\limits_{j=i+1}^{n-1} \sum\limits_{l=j+1}^{n} \frac{1}{(l-i+1)(l-i)(j-i)}$

$E[X_{i,j} X_{k,l}] = \sum\limits_{i=1}^{n-2}
\sum\limits_{j=i+1}^{n-1} \left[ \frac{1}{(j-i)} \sum\limits_{l=j+1}^{n} \frac{1}{(l-i+1)(l-i)} \right]
\leq \sum\limits_{i=1}^{n-2}
\sum\limits_{j=i+1}^{n-1} \frac{1}{(j-i)(j-i+1)}$

Hence, $E[X_{i,j} X_{k,l}] \leq n - H_{n}$.

\vspace{0.5cm}

{\bf Case 5:} Intersection between intervals $[i, j]$ and $[k, l]$ with $j = k$.

$E[X_{i,j} X_{k,l}] = \sum\limits_{i=1}^{n-2}
\sum\limits_{j=i+1}^{n-1} \sum\limits_{l=j+1}^{n} \frac{1}{(l-i+1)(l-i)(l-i-1)}
\leq \sum\limits_{i=1}^{n-2}
\sum\limits_{j=i+1}^{n-1} \sum\limits_{l=j+1}^{n} \frac{1}{(l-i+1)(l-i)}
$

$E[X_{i,j} X_{k,l}] \leq \sum\limits_{i=1}^{n-2}
\sum\limits_{j=i+1}^{n-1} \sum\limits_{l=j+1}^{n} \left( \frac{1}{l-i} - \frac{1}{l-i+1} \right)
\leq \sum\limits_{i=1}^{n-2}
\sum\limits_{j=i+1}^{n-1} \frac{1}{j-i+1}
$

$E[X_{i,j} X_{k,l}] \leq \sum\limits_{i=1}^{n-2} H_{n}$. Therefore, $E[X_{i,j} X_{k,l}] = O(n \log n)$.


{\bf Case 6:} Intersection between intervals $[i, j]$ and $[k, l]$ with $j = l$.

This is identical to case 4. Hence, $E[X_{i,j} X_{k,l}] \leq n - H_{n}$.


\vspace{0.5cm}


Once that all six cases were analised we are able to estimate $E[X^2]$.
First, it is important to observe that, with exception the case 2a, all other cases
are upper bounded by $O(n \log n)$. In the case 2a, in particular,
$E[X_{i,j} X_{k,l}] = \frac{n^2}{2} + \Theta(n \log n)$. Furthermore, by the definition of $E[X^2]$,
with exception the case 1, the results of all other cases must be multiplied by 2.
Therefore, $E[X^2] = n^2+ \Theta(n \log n)$.

Recall that, from Chebyshev's inequality, $Pr\{ |X - E[X]| \geq c E[X] \} \leq \frac{E[X^2] - E[X]^2}{c^2 E[X]^2}$.

Consequently, $Pr\{ X \geq c n \} \leq Pr\{ X \geq c (n - H_{n}) \} = \frac{n^2 + \Theta(n \log n) - (n - H_{n})^2}{c^2 (n - H_{n})^2}$.

Therefore, $Pr\{ X \geq c n \} \leq \frac{c' \log n}{c^2 n}$ where $c'$ is a positive constant.




\end{proof}

 
\end{document}





